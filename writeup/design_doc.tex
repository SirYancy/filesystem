\documentclass[12pt]{article}
\usepackage{fullpage}
\usepackage{indentfirst}
\usepackage{enumitem}
\newcommand\tab[1][1cm]{\hspace*{#1}}

\title{\large Project \#3: File System Simulator\\ \large Design Document}
\author{%
  Jared Willard
  \and  Eric Kuha}

\begin{document}

\maketitle

\paragraph{Kernel.link} 
\hfill \break
Kernel.link() takes two paths as parameters. It locates the first parameter and
acquires an instance of its inode and inode number. The second path is used to
create a new directory entry linked to the same inode creating a hard link. In
the process, the method increments the inode's \textbf{nlink} attribute before
writing it to the filesystem.

\paragraph{Kernel.unlink}
\hfill \break
The Kernel.unlink() method takes a path name as a parameter. It locates the
directory entry and node. It shifts all entries after it in the directory up
one slot. If nlink is 0, it then deallocates all blocks pointed to by the inode
and finally, it truncates the size of the target directory by one, cutting off
the last file (which was already shifted up by one). The test program, called
\textbf{rm.cc} is effectively identical in function to the basic bash command
'rm'.

\paragraph{Indirect Blocks}
\hfill \break
Indirect blocks are functionality added to the IndexNode class. It is really
quite straightforward. When the file is too large for direct block
allocation, a pointer to an array of integers is allocated with a further ten
block numbers to allocate. Functionality related to indirect blocks is added in
the constructor (where the block is initialized to an array of empty pointers),
the getBlockAddress() method which knows to look for any block number higher
than MAX\_DIRECT\_BLOCKS in the level one indirect block array, the
setBlockAddress() which works much like its respective getter, and then the
read() and write() methods which write all of the inode's information to a
buffer which is then written to or read from the disk. It's a straighforward
concept and its implementation is similarly straightforward.

\paragraph{fsck.cc}
\hfill \break
A file system checker is a critical piece of programming for several reasons.
It aids in garbage collecting orphaned disk blocks, directory entries, and
inodes (though ours doesn't bother to reclaim inodes). It ensures that all
inodes have the correct nlink value, accurately describing the number of
directory entries that point to it. Next, it verifies that all blocks mentioned
in inodes are marked as allocated blocks, and all other blocks are also marked
free.
\end{document}
